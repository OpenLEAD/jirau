
%%******************************************************************************
%% SECTION - Purpose 
 
Usually the Introduction is one paragraph that explains the objectives or purpose of the lab. In one sentence, state the hypothesis. Sometimes an introduction may contain background information, briefly summarize how the experiment was performed, state the findings of the experiment, and list the conclusions of the investigation. Even if you don't write a whole introduction, you need to state the purpose of the experiment, or why you did it. This would be where you state your hypothesis.
%%******************************************************************************

\section{Propósito}
O experimento 2, sensor indutivo, tem como principais objetivos:
 \begin{itemize}
 \item Avaliar a estrutura mecânica desenvolvida para o acoplamento do sensor
 indutivo na viga pescadora;
 \item Verificar a distância máxima de operação do sensor com o stoplog;
 \item Observar possíveis falso-positivos do sensor devido ao acoplamento
 metálico;
 \item Avaliar possíveis danos ao sensor e novas formas de acoplamento;
 \end{itemize}
\label{proposito}
O experimento consistiu em acoplar o sensor indutivo à viga utilizando a
estrutura metálica desenvolvida em laboratório pelo Prof. Ramon. O projeto
mecânico foi desenvolido a partir do desenho detalhado da viga,
disponível em CAD.
O sensor indutivo é alimentado por duas baterias de $12V$, $7Ah$, em série, formando uma alimentação de $24V$ para
os sensores. A comunicação do sensor é como um relé, isto é, a saída tem a mesma
tensão da bateria. Uma placa customizada, desenvolvida pelo grupo da eletrônica,
gerencia a alimentação das baterias ao sensor e possui um
dispositivo gateway ethernet que traduz a saída para um booleano e
disponibiliza o resultado na rede Ethernet.

Os resultados deste experimento mostraram que a escolha de um sensor indutivo
faceado e com alcance de $20mm$ é o ideal para a aplicação.
