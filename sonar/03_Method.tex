
%%******************************************************************************
%% SECTION -•	Métodos 
Describe the steps you completed during your investigation. This is your procedure. Be sufficiently detailed that anyone could read this section and duplicate your experiment. Write it as if you were giving direction for someone else to do the lab. It may be helpful to provide a Figure to diagram your experimental setup.

%%******************************************************************************

\subsection{Método}

A estrutura de fixação do sonar foi acoplada mais ao centro da viga pescadora o
quanto foi possível. Entretanto uma placa de sustentação que não estava presente
nos desenho de detalhamento que tinhamos acesso até o momento da impossibilitou
uma fixação exatamente ao centro. A figura \ref{placa_viga} FIGURA mostra a
placa soldada e a espessura extra presente nessa parte da viga pescadora. A figura
\ref{fixaca_sonar} FIGURA mostra a posição encotrada para a fixação do sonar e
sua estrutura.

Após o acoplamento da estrutura do sonar na viga pescadora, foi realizado a
conexão do cabo umbilical e sua fixação na viga pescadora, de maneira que fosse
minimizada a possibilidade de emaranhamento do cabo ou que o mesmo prendesse em
algum objeto pelo caminho, como pode ser observado nas figuras REF FIGURA.

O carretel e o computador conectado a ele devem estar posicionados de uma
maneira que possibilite um fácil enrolamento e desenrolamento do cabo, a fim de
uma correta inserção e remoção da viga pescadora no fosso %TODO verificar
% nomes
. Caso contrário pode haver um tensionamento no cabo e há risco de rompimento
das conexões.

A estrutura de fixação do sonar possibilitava dois graus de liberdade para o
ajuste da direção de varredura do sensor. A figura \ref{close_sonar} FIGURA
mostra os dois graus de liberdade, assim como as escalas para o ajuste nas
posições desejadas.

Foram realizados 3 baterias de testes com direções diferentes:  

\begin{itemize}
  \item \textbf{Primeiro teste:} ANGULOS - Direção de varredura perpendicular ao
  comprimento do fosso.
  \item \textbf{Segundo teste:} ANGULOS - Direção de varredura paralela ao
  comprimento do fosso.
  \item \textbf{Terceiro teste:} ANGULOS - Repetição do primeiro teste.
\end{itemize}

Em todos os testes foram feitas aquisições de dados na superfície, em 18,12 e 6
metros, alternando entre os sistemas Windows e Ubuntu, para a aquisição de dados com o
software da Tritech e o componente em ROCK. Cada aquisição com a viga estática
tinha a duração de pelo menos 2 minutos.

Por razões de segurança, a inserção deve ser realizada com o
software da Tritech (Windows), enquanto não houver uma visualização confiável para a aferição da distância ao fundo. No primeiro teste o operador garantiu que
existia uma marcação no cabo do pórtico e que ele pararia a viga pescadora a uma
distância segura do solo, o que não ocorreu. O primeiro teste foi repetido para
uma aquisição mais consistente de dados, devido à falta de medição confiável de
profundidade.


 

\label{metodos}


