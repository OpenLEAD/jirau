
%%******************************************************************************
%% SECTION - Materials 
List everything needed to complete your experiment.
%%******************************************************************************


%###
\subsection{Materiais}
%###
\label{materials}
Os materiais utilizados para a execução do experimento do sonar no vão do
\emph{stoplog} em montante foram:
\begin{itemize}
  \item Sonar Tritec Micron MK3 3000m;
  
  \item Computação necessária:
  \begin{itemize}
    \item Laptop com sistema operacional Ubuntu e driver ROCK instalado;
    \item Laptop com sistema operacional Windows e programa proprietário da Tritec instalado.
  \end{itemize}
  
  \item Equipamentos de integração:
  \begin{itemize}
    \item Cabo Tritec de 6 vias, com conector específico para o referido sonar,
  extendido por emenda submarina para conjunto de par trançado;
    \item fonte de 24V 6A.
    \item barra sindal, para conexão entre componentes;
  \end{itemize}
  
  \item Montagem:
  \begin{itemize}
    \item Estrutura mecânica biarticulada para fixação e direcionamento do sonar
  sob a viga pescadora;
    \item Ferramentas: chave de fenda e de canhão 8mm;
  \end{itemize}
\end{itemize}


Os sensores indutivos NBB20-L2-E2-V1 adquiridos na Pepperl-Fuchs foram
previamente testados em laboratório nas condições que se esperava encontrar em JIRAU (ver relatório XX): objetos
metálicos em torno do sensor, sensor dentro e fora da água, distância de 20 a
30mm do objeto a ser detectado. FIGURA

Os cabos M12 de 4 vias e 5m de comprimento, também adqui\-ridos na
Pepperl-Fuchs, foram estendidos com cabos que se conectam ao tubo da
eletrô\-nica embarcada. As extensões são do tipo submarina, capaz de re\-sistir a altas pressões embaixo
d'água. FIGURA

O cabo umbilical da eletrônica embarcada apresenta 12 vias, onde duas são as
saídas de sinais dos sensores indutivos. Essas saídas garantem a
verificação dos sensores indutivos caso haja falha do dispositivo GPIO/Ethernet
da placa eletrônica. A ferramenta utilizada para essa verificação é o
voltímetro.

A eletrônica embarcada do teste em JIRAU é um protótipo simplificado da
eletrônica final do projeto ROSA. Ela pode ser subdividida em um projeto
mecânico, placa eletrônica e potência. 

A estrutura mecânica da eletrônica embarcada deve atender seguintes requisitos
de projeto: imersível 100m em água, resistente à vibração, resistente a choque
mecânico e acoplamento simples à viga pescadora, isto é, sem causar alterações à
estrutura. A solução rápida e simples adotada foi a montagem e adaptação do
antigo tubo do projeto LUMA, ROV com expedição Antártida. FIGURA

A placa eletrônica customizada é responsável pela distribuição e conversão da
potência, comunicação e conversão do meio físico entre dispositivos através de
um dispositivo GPIO/Ethernet. A entrada para sensor indutivo na placa eletrônica
é um conector de 4 vias, onde apenas 3 são utilizadas: potência $24V$
diretamente da bateria ou fonte externa, aterramento (Ground da bateria ou fonte
externa) e sinal. O sinal é uma saída do tipo relé, ou seja, $24V$ quando há
proximidade faceada com metal. A saída, então $24V$, passa por um conversor de
tensão, o que garante uma queda para $3.3V$. Um dispositivo GPIO/Ethernet
converte a saída de nível lógico TTL para Ethernet. Vale ressaltar que a
conexão da placa ocorre com os conectores internos do tubo, estrutura mecânica, e não diretamente com o cabo do sensor indutivo. A lógica da montagem é: sensor indutivo - Cabo
M12 - Emenda cabo externo do tubo - Conector tubo - Cabo interno do tubo -
Conector placa - Placa. FIGURA

O computador com sistema operacional Ubuntu recebe os dados via Ethernet pelo
Modem. O sistema para obtenção dos dados foi desenvolvido em ROCK.

A potência da eletrônica embarcada é fornecida por duas baterias $12V$, $7AH$ em
série, o que garante um total de $24V$. Elas são alojadas internamente ao tubo,
podendo ser desligadas por uma chave externa ao tubo, com proteção à prova
d'água FIGURA. O projeto ainda possibilita a entrada de uma fonte externa de
alimentação por uma das saídas do umbilical, em caso de falha nas baterias.

A estrutura mecânica para acoplamento do sensor foi desenvolvida pelo prof.
Ramon e pode ser observada na FIGURA. O formato da barra acompanha a lateral
da garra pescadora, oposta à pegada com o stoplog. Na barra ortogonal à
primeira, localizada na parte inferior, é acoplado o sensor indutivo, garantindo
sua proteção em relação ao olhal do stoplog. O acoplamento à viga pescadora foi
realizado com abraçadeiras.




