
%%******************************************************************************
%% SECTION - Materials 
%% List everything needed to complete your experiment.
%%******************************************************************************


%###
\subsection{Materiais}
%###
\label{materials}
Os materiais utilizados para a execução do experimento do sonar no vão do
\emph{stoplog} em montante foram:
\begin{itemize}
  \item Sonar Tritech Micron MK3 3000m;
  
  \item Computação necessária:
  \begin{itemize}
    \item Laptop com sistema operacional Ubuntu e driver ROCK instalado;
    \item Laptop com sistema operacional Windows e programa proprietário da Tritec instalado.
  \end{itemize}
  
  \item Equipamentos de integração:
  \begin{itemize}
    \item Cabo Tritec de 6 vias, com conector específico para o referido sonar,
  extendido por emenda submarina para conjunto de par trançado;
    \item fonte de 24V 6A;
    \item conversor 232/USB;
    \item barra sindal, para conexão entre componentes.
  \end{itemize}
  
  \item Montagem:
  \begin{itemize}
    \item Estrutura mecânica biarticulada para fixação e direcionamento do sonar
  sob a viga pescadora;
    \item Ferramentas: chave de fenda e de canhão 8mm;
  \end{itemize}
\end{itemize}

O sonar Micron utilizado nos testes em Jirau já tivera seu funcionamento
avaliado em testes no tanque do LabOceano, no parque tecnológico. Na ocasiação
pode-se ver com clareza a capacidade de resolução do sonar o que tornou viável o
julgamento da qualidade da leitura durante os testes na viagem.

O cabo Tritec de 6 vias e foi extendido por cabos ethernet até um total de 40m.
A emenda é do tipo submarina, capaz de re\-sistir a pressões de até 1000m
embaixo d'água.

Os pares trançados provenentes do cabo ethernet extendido são organizados em uma
barra sindal de maneira que um par é conectado a uma fonte 24V e os outros
entram em um conversor 232/USB e que, por sua vez, é conectado ao Laptop.

Nos testes foram utilizados dois laptops. Um possuía sistema operacional Ubuntu
e software desenvolvido utilizando o framework ROCK para a coleta e registro dos
valores fornecidos pelo sonar. O outro utilizou o software proprietário da
Tritec rodando em um ambiente Windows, gerando gravações da visualização feita
apartir deste. Ambos liam os dados por USB através do conversor.


%% Daqui pra baixo.
A estrutura mecânica da eletrônica embarcada deve atender seguintes requisitos
de projeto: imersível 100m em água, resistente à vibração, resistente a choque
mecânico e acoplamento simples à viga pescadora, isto é, sem causar alterações à
estrutura. A solução rápida e simples adotada foi a montagem e adaptação do
antigo tubo do projeto LUMA, ROV com expedição Antártida. FIGURA

A estrutura mecânica para acoplamento do sensor foi desenvolvida pelo prof.
Ramon e pode ser observada na FIGURA. O formato da barra acompanha a lateral
da garra pescadora, oposta à pegada com o stoplog. Na barra ortogonal à
primeira, localizada na parte inferior, é acoplado o sensor indutivo, garantindo
sua proteção em relação ao olhal do stoplog. O acoplamento à viga pescadora foi
realizado com abraçadeiras.




