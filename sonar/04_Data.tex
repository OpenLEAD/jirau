
%%******************************************************************************
%% SECTION - Data 

%Numerical data obtained from your procedure usually is presented as a table.
% Data encompasses what you recorded when you conducted the experiment. It's just the facts, not any interpretation of what they mean.

%%******************************************************************************

\subsection{Dados}
\label{dados}

Os dados adquiridos consistem em um conjunto de ecos recebidos pelo sonar
contendo, cada um, uma informação de intensidade e direção. A partir das
características do sonar utilizado no teste, cada eco recebido é mapeado em um
\textit{Sonar Beam} leque que é subdividos em comprimentos fixos, denomindados
\textit{bins}. A figura %TODO figura 
ilustra a visualização de um sonar beam.

A partir da combinação de todos os sonar beams aquisitados em uma bateria de
testes é possível gerar uma visualização primitiva do ambiente. Tanto no sistema
Windows, quanto no Ubuntu, só é possível gerar imagens contendo informações em
um único plano. Mesmo no Ubuntu, onde já se se extrapola a visualização para o
espaço tridimensional, só é possível extrair conteúdo em 2D, devido a ausência
de uma unidade Pan-Tilt para a movimentação do sonar e a mudança de sua direção.

Foram aquisitados os seguintes conjuntos de dados:

\begin{itemize}
  \item Ângulo do sonar = 90
  \begin{itemize}
    \item Superfície
    \item 12 metros
    \item 6 metros
    \item Subida
  \end{itemize}
  \item Angulo do sonar = 180
  \begin{itemize}
    \item Superfície
    \item 12 metros
    \item 6 metros
    \item Subida
  \end{itemize}
\end{itemize}


  