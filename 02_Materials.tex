
%%******************************************************************************
%% SECTION - Materials 
List everything needed to complete your experiment.
%%******************************************************************************



\section{Materiais}
Os materiais utilizados para a execução do experimento do sensor indutivo na
Usina foram:
\begin{itemize}
  \item Três sensores indutivos;
  \item Cabos M12 com conectores submarinos;
  \item Eletrônica embarcada à prova d'água composta por: tubo metálico, duas
  baterias, chave de ativação, placa eletrônica customizada e cabeamento;
  \item Estrutura mecânica para acoplamento do sensor;
\end{itemize}
Os sensores indutivos NBB20-L2-E2-V1 adquiridos na Pepperl-Fuchs foram
previamente testados em laboratório nas condições que se esperava encontrar em JIRAU (ver relatório XX): objetos
metálicos em torno do sensor, sensor dentro e fora da água, distância de 20 a
30mm do objeto a ser detectado. FIGURA

Os cabos M12 de 4 vias e 5m de comprimento, também adquiridos na Pepperl-Fuchs,
foram estendidos com cabos que se conectam ao tubo da eletrônica embarcada. As
extensões foram do tipo submarina, capaz de resistir a altas pressões embaixo
d'água. FIGURA

\label{materials}




