
%%******************************************************************************
%% SECTION -•	Métodos 
Describe the steps you completed during your investigation. This is your procedure. Be sufficiently detailed that anyone could read this section and duplicate your experiment. Write it as if you were giving direction for someone else to do the lab. It may be helpful to provide a Figure to diagram your experimental setup.

%%******************************************************************************

\section{Métodos}
Esta seção está subdividida em: Logística de materiais e dispositivos,
planejamento do experimento, montagem, execução e coleta de dados,
desmontagem e fechamento.

\subsection{Logística de materiais e dispositivos}
Esta subseção abrange tanto pesquisa e escolha de malas/cases para
transporte de equipamentos e ferramentas, quanto organização interna dos
cases, proteção para transporte aéreo, meios de locomoção e dificuldades
encontradas.

A pesquisa e compra do pelican-case para transporte de ferramentas foi
realizada duas semanas antes da viagem, porém a alteração no escopo dos experimentos, na mesma
semana, exigiu uma nova pesquisa. Os professores Ramon e Jacoud avaliaram o
tempo ainda disponível e consideraram a viagem uma boa oportunidade para testar
os novos sensores que foram entregues:
profundímetro da Velki e sensor inercial (IMU), dos projetos LUMA e DORIS. Este
último faria o papel do inclinômetro no escopo do projeto. 

Devido ao teste extra, houve a necessidade de um novo projeto para a eletrônica
embarcada à prova d'água, cabos com emendas submarinas e montagem da estrutura mecânica, além da readaptação da placa eletrônica para os novos sensores com acréscimos de novos
CIs e uma grande reestruturação do software. A compra de novos componentes,
cases e cabos foram no centro da cidade do Rio de Janeiro, Rua República do
Líbano, pelo método de reembolso e com transporte particular.

O umbilical proposto para o novo teste da eletrônica é composto por um cabo
emborrachado de 12 vias com 40m de comprimento e um cabo Ethernet com 8 vias,
sem carretel.
Os outros diversos cabos da eletrônica, fontes, baterias, voltímetro,
osciloscópio, dentre outros equipamentos, exigiram a aquisição de um case
$65x65x65cm$ com rodas, totalizando um peso de $70kg$. Não foi possível realizar
a organização interna do case, já que o tempo da última semana foi reservado, em
sua maior parte, para a reestruturação eda letrônica e de software a fim de
garantir o último teste e a obtenção de dados.

O tubo à prova d'água que contém a eletrônica embarcada com profundímetro e
IMU foi enrolada em espuma, presa com abraçadeiras de plástico. No aeroporto,
foi necessário envolvê-lo com proteção adicional. A estrutura metálica do sensor
indutivo foi revestida com plástico bolha e também envolvido com proteção
adicional.

O transporte dos cases foram realizados do laboratório à Usina nas
seguintes etapas:
\begin{itemize}
  \item Laboratório-Aeroporto: carro particular;
  \item Aeroporto: despachados;
  \item Aeroporto-Nova Mutum Paraná: carro alugado Hilux;
  \item Nova Mutum Paraná-Usina: carro alugado Hilux;
\end{itemize} 

Os três sensores indutivos foram transportados em suas respectivas caixas, em
mochila particular durante todo o trajeto da viagem.

Podemos destacar algumas dificuldades de logística:
\begin{itemize}
  \item Transporte do case de $70 Kg$: apesar de apresentar rodinhas, o
  constante deslocamento do case entre carros era complicado e consumia tempo.
  \item Não houve organização do pelican-case e do case KGB $70 Kg$. A
  necessidade de uma ferramenta ou equipamento tomava tempo pela busca e poderia
  até resultar na desmontagem do case para se ter acesso a um equipamento que
  estivesse no fundo.
  \item Para o tubo, não foi projetado um case personalizado para transporte, o
  que resultou em uma proteção improvisada com espumas, e não facilitou a sua locomoção. 
\end{itemize} 
 
\subsection{Planejamento}
A realização de um teste em campo exige o planejamento de dispositivos
necessários para sua execução levando em consideração voltagem disponível na
USINA, equipamentos que podem ser fornecidos pela USINA, operários disponíveis
para realizar a operação, tempo de uso da viga e do ambiente.

Os dispositivos e materiais para a realização do teste do sensor indutivo podem
ser observados no capítulo Materiais e foram todos disponibilizados pelo
laboratório LEAD, com exceção do Modem Ethernet. A eletrônica embarcada foi
alimentada por duas baterias, assim como os sensores, os notebooks possuem
baterias próprias, portanto a única alimentação externa necessária é do Modem
220VAC, disponível na USINA.

As vigas foram utilizadas em todos os testes, portanto um operador foi
requisitado para posicionar a viga quando necessária, com dois ajudantes que
verificavam a inclinação da viga.

O tempo do teste foi dividido em tempo de montagem e coleta de dados. Cerca de
quatro horas foram reservados para a montagem e três para a coleta dos dados.
O operador só é necessário durante a coleta de dados e na primeira fase de
montagem para posicionar a viga.

\subsection{Montagem}
A primeira fase da montagem é a separação dos materiais para o teste, dos cases
de ferramentas e equipamentos para a tenda disponibilizada pela USINA e o
pórtico, como bancada.

O operador realizou a pegada do stoplog com ajudantes e desceu o conjunto
stoplog/viga no fosso até que as garras ficassem na altura para montagem,
FIGURA.
As grades de proteção foram retiradas para o início da montagem.

O grupo se vestiu com o cinturão de segurança e deu prosseguimento a
montagem. A estrutura mecânica para o sensor indutivo e o
tubo da eletrônica embarcada foram acoplados à viga por meio de abraçadeiras
metálicas, observe FIGURA. O segundo sensor indutivo, garra esquerda, não pôde
ser acoplado, já que a estrutura mecânica foi projetada de forma simétrica e a
viga apresenta uma barra rígida entre as garras, não simétrico.

Os cabos dos sensores foram presos à viga com abraçadeiras de plástico e o
umbilical enrolado para o lado de fora do fosso para a fase de inserção.



\subsection{Execução e coleta de dados}
\subsection{Desmontagem e fechamento}

\label{metodos}


