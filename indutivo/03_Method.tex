
%%******************************************************************************
%% SECTION -•	Métodos 
Describe the steps you completed during your investigation. This is your procedure. Be sufficiently detailed that anyone could read this section and duplicate your experiment. Write it as if you were giving direction for someone else to do the lab. It may be helpful to provide a Figure to diagram your experimental setup.

%%******************************************************************************

\subsection{Métodos}
Esta seção está subdividida em: Logística de materiais e dispositivos,
planejamento do experimento, montagem, execução e coleta de dados,
desmontagem e fechamento.
 
\subsubsection{Planejamento}

Os dispositivos e materiais para a realização do teste do sensor indutivo podem
ser observados no capítulo Materiais e foram todos disponibilizados pelo
laboratório LEAD, com exceção do Modem Ethernet. A eletrônica embarcada foi
alimentada por duas baterias, assim como os sensores, os notebooks possuem
baterias próprias, portanto a única alimentação externa necessária é do Modem
220VAC, disponível na USINA.

As vigas foram utilizadas em todos os testes, portanto um operador foi
requisitado para posicionar a viga quando necessária, com dois ajudantes que
verificavam a inclinação da viga.

O tempo do teste foi dividido em tempo de montagem e coleta de dados. Cerca de
quatro horas foram reservados para a montagem e três para a coleta dos dados.
O operador só é necessário durante a coleta de dados e na primeira fase de
montagem para posicionar a viga.

\subsubsection{Montagem}
A primeira fase da montagem é a separação dos materiais para o teste, dos cases
de ferramentas e equipamentos para a tenda disponibilizada pela USINA e o
pórtico, como bancada.

O operador realizou a pegada do stoplog com ajudantes e desceu o conjunto
stoplog/viga no fosso até que as garras ficassem na altura para montagem,
FIGURA.
As grades de proteção foram retiradas para o início da montagem.

O grupo se vestiu com o cinturão de segurança e deu prosseguimento a
montagem. A estrutura mecânica para o sensor indutivo e o
tubo da eletrônica embarcada foram acoplados à viga por meio de abraçadeiras
metálicas, observe FIGURA. O segundo sensor indutivo, garra esquerda, não pôde
ser acoplado, já que a estrutura mecânica foi projetada de forma simétrica e a
viga apresenta uma barra rígida entre as garras, não simétrico.

Os cabos dos sensores foram presos à viga com abraçadeiras de plástico e o
umbilical enrolado para o lado de fora do fosso para a fase de inserção.



\subsubsection{Execução e coleta de dados}
\subsubsection{Desmontagem e fechamento}

\label{metodos}


